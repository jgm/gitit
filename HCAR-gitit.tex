\begin{hcarentry}[updated]{gitit}
\label{gitit}
\report{John MacFarlane}%11/08
\participants{Gwern Branwen, Simon Michael, Henry Laxen, Anton
van Straaten, Robin Green, Thomas Hartman, Justin Bogner, Kohei Ozaki}
\status{active development}
\makeheader

Gitit is a wiki program written in Haskell. It uses~\cref{Happstack}
for the web server and~\cref{Pandoc} for markup processing. Pages and
uploaded files are stored in a Git or~\cref{Darcs} repository and may
be modified either by using the VCS's command-line tools or through the
wiki's web interface. By default, Pandoc's extended version of markdown
is used as a markup language, but reStructuredText, LaTeX, or
HTML can also be used.
Pages can be exported in a number of different formats, including LaTeX,
RTF, OpenOffice ODT, and MediaWiki markup. Gitit can be configured to
display TeX math (using jsMath) and highlighted source code
(using~\cref{highlighting-kate}).

Gitit is currently being used by a number of people and projects, such
as~\cref{LHC} and Darcs.

Gitit features:
\begin{itemize}
\item Distributed wiki
\item Automatic syntax highlighting of source code files
\item Export pages in any format supported by pandoc
\item Customizable through plugins that can transform pages at
  the AST level
\end{itemize}

We are currently working on the following:
\begin{itemize}
\item Literate Haskell support
\item RSS feeds for pages
\item Using new Happstack architecture
\end{itemize}

\FurtherReading
\begin{itemize}
\item Documentation and source: \text{\url{http://github.com/jgm/gitit/tree/master}} 
\item HackageDB: \texttt{cabal install}
 \text{\url{http://hackage.haskell.org/cgi-bin/hackage-scripts/package/gitit}}
\item Demo: \text{\url{http://gitit.johnmacfarlane.net}}
\end{itemize}
\end{hcarentry}
