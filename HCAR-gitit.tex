\begin{hcarentry}[updated]{gitit}
\label{gitit}
\report{John MacFarlane}%11/10
\participants{Gwern Branwen, Simon Michael, Henry Laxen, Anton
van Straaten, Robin Green, Thomas Hartman, Justin Bogner, Kohei Ozaki,
Dmitry Golubovsky, Anton Tayanovskyy, Dan Cook}
\status{active development}
\makeheader

Gitit is a wiki built on \cref{Happstack} and backed by a git or darcs
filestore. Pages and uploaded files can be modified either directly
via the VCS's command-line tools or through the wiki's web interface.
\cref{Pandoc} is used for markup processing, so pages may be written in
(extended) markdown, reStructuredText, LaTeX, HTML, or literate Haskell,
and exported in ten different formats, including LaTeX, ConTeXt,
DocBook, RTF, OpenOffice ODT, and MediaWiki markup.

Notable features of gitit include
\begin{itemize}
\item
  Plugins: users can write their own dynamically loaded page transformations,
  which operate directly on the abstract syntax tree.
\item
  Math support:  LaTeX inline and display math is automatically converted
  to MathML, using the \texttt{texmath} library.
\item
  Highlighting:  Any git or darcs repository can be made a gitit wiki.
  Directories can be browsed, and source code files are
  automatically syntax-highlighted.  Code snippets in wiki pages
  can also be highlighted.
\item
  Library: Gitit now exports a library, \texttt{Network.Gitit}, that makes it
  easy to include a gitit wiki (or wikis) in any \cref{Happstack} application.
\item
  Literate Haskell: Pages can be written directly in literate Haskell.
\end{itemize}

\FurtherReading
For more information, see the gitit home page, which is itself
a running demo of gitit:  \text{\url{http://gitit.net}}.
\end{hcarentry}
